\documentclass{sig-alternate}

\usepackage{color}
\usepackage{hyperref}
\hypersetup{colorlinks=true,citecolor=green, linkcolor=red}


%\usepackage[vlined,algoruled,titlenumbered,noend]{algorithm2e}
\usepackage{amsmath,amsfonts,amssymb} % ,amsthm}
\usepackage{array}
\usepackage{amsmath,amssymb}
\usepackage{epsfig,subfigure}
\usepackage{pgfplots}
\usepackage{verbatim}

\newcommand{\fix}{\marginpar{FIX}}
\newcommand{\new}{\marginpar{NEW}}
\newcommand{\ind}[1]{\mathbb{I}[#1]}
\newcommand{\inde}{\mathbb{I}}

\newcommand{\var}{v}
\newcommand{\eq}{\leftarrow}

\newcommand{\LB}{\mathit{LB}}
\newcommand{\UB}{\mathit{UB}}

\newcommand{\B}{\mathbb{B}}
\newcommand{\E}{\mathbb{E}}
\newcommand{\I}{\mathbb{I}}
\newcommand{\R}{\mathbb{R}}
\renewcommand{\vec}[1]{\mathbf{#1}}

%\usepackage{aaai}
\usepackage{times}
\usepackage{helvet}
\usepackage{courier}
\frenchspacing

%\long\def\COMMENT#1\ENDCOMMENT{\message{(Commented text...)}\par}



\begin{document}

\title{Improving LDA Topic Models for Microblogs\\ via Novel Tweet Pooling Schemes}

\author{
\alignauthor
Author 1\\
       \affaddr{Affiliation 1}\\
       \affaddr{Address 1}\\
       \affaddr{Country 1}\\
       \email{email1}
% 2nd. author
\alignauthor
Author 1\\
       \affaddr{Affiliation 2}\\
       \affaddr{Address 2}\\
       \affaddr{Country 2}\\
       \email{email2}
       % 3rd. author
\alignauthor
Author 3\\
       \affaddr{Affiliation 3}\\
       \affaddr{Address 3}\\
       \affaddr{Country 3}\\
       \email{email3}
}


\maketitle
\begin{abstract}
Twitter: the world of 140 characters poses serious challenges to the
efficacy of topic models on short, messy text. While topic models such
as Latent Dirichlet Allocation (LDA) have a long history of successful
application to news articles and academic abstracts, they are often
less coherent when applied to microblog content like Twitter.  In this
paper, we investigate methods to improve topics learned from Twitter
content \emph{without} modifying the basic machinery of LDA; we
achieve this through various pooling schemes that aggregate tweets in
a data preprocessing step for LDA.  We empirically establish that
tweet pooling by hashtags leads to a vast improvement in a variety of
measures of topic coherence across three diverse Twitter datasets in
comparison to an unmodified LDA baseline and a variety of pooling
schemes.
% TODO: edit last sentence?
\begin{comment}a novel method of combining
  automatic hashtag labeling techniques with 
\end{comment}
\end{abstract}


\category{H.4}{Information Systems Applications}{Miscellaneous}

\category{D.2.8}{Software Engineering}{Metrics}[complexity measures, performance measures]

\terms{Theory}

\keywords{ACM proceedings, \LaTeX, text tagging}

\section{Introduction}
\label{sec:intro}

The ``undirected informational'' search task, where people seek to
better understand the information available in document corpora, uses
techniques such as multidocument summarisation and topic modeling.
Topic models uncover the salient patterns of a collection under the
mixed-membership assumption: each document can exhibit multiple
patterns to different extents.  When analysing text, these patterns
are represented as distributions over words, called \textit{topics}.
Probabilistic topic models such as Latent Dirichlet Allocation (LDA)
\cite{blei03} are a class of Bayesian latent variable models that have
been adapted to model a diverse range of document genres.

To address the undirected informational task arising for the
exploration of Twitter content, we propose the use of popular topic
models like LDA.  However, Twitter content poses unique challenges
different to much of standard NLP content:
(1) posts are short (140 characters or less),
(2) mixed with contextual clues such as URLs, tags, and Twitter names, and
(3) use informal language with misspelling, acronyms and 
nonstandard abbreviations (e.g. O o haha wow).
Hence, effectively modeling content on Twitter requires techniques
that can readily adapt to this unwieldy data while requiring little
supervision.

Unfortunately, it has been found that topic modeling techniques like
LDA do \emph{not} work well with the messy form of Twitter
content~\cite{wayne}.  Topics learned from LDA are formally a
multinomial distribution over words, and by convention the top-10
words are used to identify the subject area or give an interpretation
of a topic.  The na\"{i}ve application of LDA to Twitter content
produces mostly incoherent topics --- some are vaguely interpretable
but contain unrelated words in the top-10 word set.  For example,
Table~\ref{tbl-0} demonstrates poor topic words as compared to topic
words which are much more coherent and interpretable.

%%%%%%%%%%%%%%%%%%%%%%%%%%%%%%%%%%%%%%%%%%%%%%%%%%%%%%%%%%%%%%%%%%%%
\begin{table}[t!]
\centering
\caption{Sample Topic Words}\label{tbl-0}
\resizebox{8.5cm}{!} 
{
	%\begin{tabular}{|p{2in}|p{2in}|}
	\begin{tabular}{|c|c|}
	\hline
        Poor Topics  & Coherent Topics \\
\hline
 {\small barack cool apple health iphone}
 &
 {\small flu swine news pandemic health}\\
 {\small los barackobama video uma gop} & 
{\small death flight h1n1 vaccine confirmed} \\
 \hline
	\end{tabular}
}\vspace*{-10pt}
\end{table}
%%%%%%%%%%%%%%%%%%%%%%%%%%%%%%%%%%%%%%%%%%%%%%%%%%%%%%%%%%%%%%%%%%%%

How can we extract better topics in microblogging environments with
standard LDA?  While lingistic ``cleaning'' of text could help
somewhat, for instance \cite{Han2012}, a complementary approach using
LDA is also needed because there are so few words in a tweet.  An
intuitive solution to this problem is tweet
pooling~\cite{Weng2010wsdm,hong}: merging related tweets together and
presenting them as a single document to the LDA model.  In this paper
we examine various tweet-pooling schemes to improve LDA topic quality.
We compare the performance of these methods across three datasets
constructed to be representative of the diverse collections of content
possible in the microblog environment and examine a variety of topic
coherence evaluation metrics including the ability of the learned LDA
topics to reconstruct known clusters and the interpretability of
these topics via statistical information measures.

{\bf TODO:} In this paper, we make a few important contributions: 
\begin{enumerate}
\item[(1)] Hashtag pooling.
\item[(2)] Automatic labeling for improved performance.
\end{enumerate}

%%%%%%%%%%%%%%%%%%%%%%%%%%%%%%%%%%%%%%%%%%%%%%%%%%%%%%%%%%%%%%%%%%%%%%%
%%%%%%%%%%%%%%%%%%%%%%%%%%%%%%%%%%%%%%%%%%%%%%%%%%%%%%%%%%%%%%%%%%%%%%%
%%%%%%%%%%%%%%%%%%%%%%%%%%%%%%%%%%%%%%%%%%%%%%%%%%%%%%%%%%%%%%%%%%%%%%%

\section{Tweet Pooling for Topic Models}

The goal of this paper is to obtain better LDA topics from Twitter
content without modifying the basic machinery of standard LDA.  As
noted in Section~\ref{sec:intro}, microblog messages differ from
conventional text: message quality varies greatly, from newswire-like
utterances to babble.  To address these challenges with topic
modelling, we present various pooling schemes to aggregate tweets into
``macro-documents'' for use as training data to build better LDA
models.  The motivation behind tweet pooling is that individual tweets
are very short ($\leq$ 140 characters) and hence treating each tweet
as an individual document does not present adequate term co-occurence
data within documents.  Aggregating tweets which are similar in some
sense (semantically, temporally, etc.) enriches the content present in
a single document from which the LDA can learn a better topic model.
We next describe various tweet pooling schemes.

\noindent {\bf Basic scheme -- Unpooled Tweets:} The default way of
training models involves treating each tweet as a single document and
training LDA on all tweets. This serves as our baseline for comparison
to pooled schemes.

\noindent {\bf Author-wise Pooling:} Pooling tweets according to
author is a standard away of aggregating Twitter data to improve LDA
topic modeling~\cite{Weng2010wsdm,hong} and shown to be superior to
unpooled Tweets.  To use this method, we build a document for each
author which combines all tweets they have posted.

\noindent {\bf Burst-score wise Pooling:} A \textit{trend} on
Twitter~\cite{mor} (sometimes referred to as a trending topic)
consists of one or more terms and a time period, such that the volume
of messages posted for the terms in the time period exceeds some
expected level of activity.  In order to identify trends in Twitter
posts, unusual "bursts" of term frequency can be detected in the data.
We run a simple burst detection algorithm to detect such trending
topics and aggregate tweets containing those terms having high burst
scores.  To identify terms that appear more frequently than expected,
we will assign a score to terms according to their deviation from an
expected frequency. Assume that $M$ is the set of all messages in our
tweets dataset, $R$ is a set of one or more terms (a potential
trending topic) to which we wish to assign a score, and $d \in D$
represents one day in a set $D$ of days.  We then define $M(R, d)$ as
the subset of Twitter messages in $M$ such that (1) the message
contains all the terms in $R$ and (2) the message was posted during
day $d$.  With this information, we can compare the volume in a
specific day to the other days.  Let $\mathit{Mean}(R) = \frac{1}{|D|}
\Sigma_{d \in D} M(R,d)$.  Correspondingly, $\mathit{SD}(R)$ is the
standard deviation of $M(R,d)$ over the days $d \in D$.  The
\textit{burst-score} is then defined as:
\[
\mathit{burst\textrm{-}score}(R,d) = \frac{|M(R,d) - \mathit{Mean}(R)|}{\mathit{SD}(R)} 
\]
Let us denote an individual term having burst-score greater than some
threshold $\tau$ on some day $d \in D$ as a \textit{burst-term}.  Then
our first novel aggregation method of Burst Score-wise Pooling
aggregates tweets for each burst-term into a single document for
training LDA, where we found $\tau = 5$ to provide best results.

\noindent {\bf Temporal Pooling:} When a major event occurs, a
large number of users often start tweeting about the event within a
short period of time.  To capture such temporal coherence of tweets,
the fourth scheme and our second novel pooling proposal is known as
Temporal Pooling, where we pool all tweets posted within the same
hour.

\noindent {\bf Hashtag-based Pooling:} A Twitter \textit{hashtag} is a
string of characters preceded by the hash (\#) character. In many
cases hashtags can be viewed as topical markers, an indication to the
context of the tweet or as the core idea expressed in the tweet,
therefore hashtags are adopted by other users that contribute similar
content or express a related idea. Two examples of the use of hashtags
are "ask GAGA anything using the tag \#GoogleGoesGaga for her
interview! RT so every monster learns about it!! " referring to an
exclusive interview for Google by Lady Gaga (singer) and "Whoever said
'youth is wasted on the young' must be eating his words right
now. \#March15 \#Jan25 \#Feb14 ", referring to the protest movements
in the Arab world.  For the hashtag-based pooling scheme, for each
create pooled documents for each hashtag. If any tweet has more than
one hashtag, this tweet gets added to the tweet-pool of each of those
hashtags.

\noindent {\bf Other Pooling:} While a few other combinations of
pooling schemes (eg.author-time, hashtag-time, \textit{etc}) are
possible, the initial results obtained were not as good as those
presented for the currently outlined pooling schemes.  This may be due
to the lack of data in each finer-grained pool.  Despite the initial
negative results, these combinations of pooling schemes might be
further explored in future work and may help unveil even finer-grained
topics (i.e., short-term events centred on an author group or set of
hashtags).

%%%%%%%%%%%%%%%%%%%%%%%%%%%%%%%%%%%%%%%%%%%%%%%%%%%%%%%%%%%%%%%%%%%%%%%
%%%%%%%%%%%%%%%%%%%%%%%%%%%%%%%%%%%%%%%%%%%%%%%%%%%%%%%%%%%%%%%%%%%%%%%
%%%%%%%%%%%%%%%%%%%%%%%%%%%%%%%%%%%%%%%%%%%%%%%%%%%%%%%%%%%%%%%%%%%%%%%

\section{Twitter Dataset Construction}

\label{sec:dataset}

%The different pooling schemes and their proposed modifications result
%in different topic models, the evaluation of which is a major
%concern. We wish to answer questions like: Which scheme performs
%better on which aspects and on what kinds of data? Due to the large
%number of tweets ($\sim$360K) in any of the twitter specific
%datasets, manual labeling of topics is not feasible.  To circumvent
%this problem of unsupervised evaluation we carefully construct our
%datasets keeping the following point in mind: The datasets should
%cover diverse collections of content, but also the known source of
%the content should help in evaluation of the different schemes.

We construct three datasets representative of the diverse collections
of content found on Twitter.  We chose one or two term queries (often
with similar pairs of queries to encourage a non-strongly diagonal
confusion matrix) to search a tweet collection and each resulting set
of tweets was labeled by the query that retrieved it.  Since the
number of queries (equivalently the number of clusters) is known
beforehand, we could use this knowledge to evaluate how well the
topics output by LDA match with known clusters. A brief description of
the three datasets is as follows:

\begin{description}
\item[Generic Dataset: ] 359478 tweets from 11 Jan'09 to 30 Jan'09.  A
  general dataset with tweets containing generic terms.\vspace{-5pt}
\item[Specific Dataset: ] 214580 tweets from 11 Jan'09 to 30 Jan'09.
  A dataset composed of tweets which have specific terms that refer to
  specific named entities.\vspace{-5pt}
\item[Event Dataset: ] 207128 tweets from 1 Jun'09 to 30 Jun'09.  A
  dataset composed of tweets pertaining to specific events.  The query
  terms represent these events and the time period was chosen
  specifically due to the number of co-occurring events being
  discussed at this time.
\end{description}

%Each of these datasets was created by querying a collection of 100
%million tweets spanning two months (Jan'09 \& Jun'09) with terms
%that relate to generic queries (broad topic words like music,
%business, {\it etc.}), specific queries (named entity topics like
%Obama, McDonalds, {\it etc.}) and event related queries (actual events
%in that timeframe like recession, Flight 447, Iran elections, {\it
%  etc.}).  
Table~\ref{tbl-q} provides the exact query terms and the percentage of
tweets in the datasets retrieved by each query.  Typically, less than
one percent of tweets were retrieved by more than one query with the
highest case of 4.6\% overlap occurring in the \emph{generic dataset}
for the two queries ``family'' and ``fun''.  We have removed tweets
retrieved by more than one query in a dataset in order to preserve
uniqueness of tweet labels for later analysis with clustering metrics.

\begin{table}%[!ht]
\centering
\caption{Datasets}\label{tbl-q}
\resizebox{8.5cm}{!} 
{
	\begin{tabular}{|c|p{4in}|}
	\hline
        Dataset & Term/\% \\
\hline
Generic &{\small music/17.9 business/15.8 movie/14.5 design/10.8
       food/9.6 fun/9.1 health/6.9 family/6.4 sport/4.9 space/3.2}  \\
%music & business & movie & design & food & fun & health & family & sport & space
%\# tw & 121511 & 107422 & 98496 & 73422 & 64723 & 61776 & 47209 & 43705 & 33758 & 24236 \\
%\% tw & 17.9 & 15.8 & 14.5 & 10.8 & 9.6 & 9.1 & 6.9 & 6.4 & 4.9 & 3.2 \\
Specific &{\small 
Obama/23.2 Sarkozy/0.4 baseball/3.5 cricket/1.8 Mcdonalds/1.5 Burgerking/0.5 Apple/16.3 Microsoft/6.8 United-states/40.7 France/4.9} \\
% Term & obama & sarkozy & baseball & cricket & mcdonalds & burgerkings & apple & microsoft & united statess & france\\
%\# tw & 96810 & 1831 & 14343 & 7627 & 6313 & 2224 & 67886 & 28497 & 169396 & 20502 \\
% tw & 23.2 & 0.4 & 3.5 & 1.8 & 1.5 & 0.5 & 16.3 & 6.8 & 40.7 & 4.9 \\
Events &{\small Flight-447/0.9 Jackson/13.9  Lakers/13.8 attack/13.8 scandal/4.1 swine-flu/13.8 recession/12.3 conference/14.1 T20/4.4 Iran-election/8.6  }\\
% Term & Flight 447 & Jackson & Lakers & attack & scandal & swine flu & recession & conference & T20 & Iran election \\
%  tw & 0.9 & 13.9 & 13.8 & 13.8 & 4.1 & 13.8 & 12.3 & 14.1 & 4.4 & 8.6 \\
	\hline
	\end{tabular}
}\vspace*{-10pt}
\end{table}
 
% We next present the evaluation metrics used to compare the different topic models learnt by the different pooling schemes. \\

%%%%%%%%%%%%

\section{Evaluation Metrics used}

\label{sec:evaluation}

%Evaluation of the different topic models based on the features of
%coherence: topical consistency of documents assigned to a topic with
%high probability, or human interpretability of the most probable words
%for a topic are both important issues, but the unsupervised nature of
%topic models makes this difficult. For some applications there may be
%extrinsic tasks, such as information retrieval or document
%classification, for which performance can be evaluated. However, such
%tasks are not applicable for evaluating topics models in the {\it
%  undirected informational task}.

Because there is no single method for evaluating topic models, we
evaluate a range of metrics including those used in clustering (purity
and NMI), semantic topic coherence and interpretability (PMI), and a
pure probabilistic approach (held-out probability)~\cite{wallach}.

{\bf SCOTT CURRENTLY CORRECTING}

\paragraph{Clustering-based metrics} 
In order to cluster with LDA, we let a topic represent each cluster
and assign each tweet to its corresponding mixture topic of highest
probability (an inferred quantity via LDA).  Then by analysing
clustering-based metrics, we wish to understand how well the different
tweet pooling schemes are able to cluster according to the original
queries used to produce the datasets.  

Formally, let $T_{i}$ be the set of tweets in LDA topic cluster $i$ and
$Q_{j}$ be the set of tweets with query label $j$.  Then let $T = \lbrace
T_{1}, \ldots , T_{|T|} \rbrace$ be the set of all $|T|$ clusters and $Q =
\lbrace Q_{1}, \ldots , Q_{|Q|} \rbrace$ be the set of all $|Q|$
query labels.  Now we define our clustering-based metrics as follows.

\noindent \textbf{Purity:} To compute purity \cite{MRS08}, each LDA
topic cluster is assigned the \emph{query label most frequent in the
cluster}.  Purity then simply measures the average ``purity'' of each
cluster, i.e., the fraction of tweets in a cluster having the assigned
cluster query label.  Obviously, high
purity scores reflect better original cluster reconstruction.


\[
 \mathit{purity}(T,G) = \frac{1}{N} \Sigma_{i \in \{ 1\ldots|T|\} } max_{j \in \{1\ldots|Q| \} } |t_{k} \cap g_{j}|
\]
where $ t_k = \lbrace  d \hspace{2 mm} |  \hspace{2 mm} argmax_{t^*} \theta^{t^{*}}_d = t \rbrace $.
As the number of correctly assigned tweets increases for each cluster,
the overall purity score increases. hence high purity scores reflect
better cluster reconstruction, hence a topic model with high purity
score is considered better.

\subsection{Normalized Mutual Information}

As a more information-theoretic measure of cluster quality, we also
evaluate normalized mutual information (NMI)~\cite{MRS08}.


Since we know the ground truth label of all the tweets in the dataset,
i.e., their categories, we can measure the quality by how likely the
topics agree with the true category labels.
But high agreement is easy to achieve when the number of clusters is
large, thus one needs a divisor to discount for a large number of clusters.
The resulting two-part score is:
\[
NMI(T,G) = \frac{2 I(T;G)}{H(T) + H(G)} 
\]
where $I(T,G)$ is Mutual Information and $H(T)$ gives the entropy. The 
corresponding values are:\vspace{-3pt}
\[
I(T,G) = \Sigma_{k} \Sigma_{j} \frac{|t_{k} \cap g_{j}|}{N} log \frac{|t_{k} \cap g_{j}|}{|t_{k}| |g_{j}|} 
~~~~~~~~~~~~~~~~~~
H(T) = - \Sigma_k \frac{|t_k|}{N} log \frac{|t_k|}{N} 
\]
NMI \cite{MRS08} is always a number between 0 and 1. NMI score will be
1 if the clustering results exactly match the category labels while 0
if the two sets are independent. For each tweet $d$, we use the
maximum value in topic mixture $ \theta_{d} $ to determine its
cluster. After this mapping process, we compute NMI scores with the
labels.

{\bf SCOTT HAS NOT EDITED BEYOND}

\noindent \textbf{Normalized Mutual Information (NMI):} Purity
measures how well the quality of LDA topics agrees with the true
category labels.  But high agreement is easy to achieve when the
number of clusters is large, thus one needs a divisor to discount for
a large number of clusters.  The resulting two-part score is:
\[
NMI(T,G) = \frac{2 I(T;G)}{H(T) + H(G)} 
\]
where $I(T,G)$ is Mutual Information and $H(T)$ gives the entropy.
NMI \cite{MRS08} is always a number between 0 and 1. NMI score will be
1 if the clustering results exactly match the category labels while 0
if the two sets are independent. 

\paragraph{Semantic coherence and interpretability}
Learnt topics should be coherent and interpretable.  Topic coherence 
meaning semantic coherence - is a human judged quality that depends on
the semantics of the words, and cannot be measured by model-based
statistical measures that treat the words as exchangeable tokens.  It
is possible to automatically measure topic coherence with near-human
accuracy \cite{baldwin10} using a score based on pointwise mutual
information (PMI).  We use this to measure coherence of the topics
from different tweet-pooling schemes.

\noindent \textbf{Pointwise Mutual Information (PMI):} One of the goals of
our work is to get topics that are more coherent.  PMI is one measure
of the statistical independence of observing two words in close
proximity.  We treat two words as co-occurring if both the words occur
in the same tweet.  For a topic $t_k$, we measure topic coherence as
the average of PMI for the pairs of its top ten words
$\{w_1,...,w_{10}\}$.
\begin{equation*}
	PMI \hspace{1.5mm} Score(t_k) = \frac{1}{100}\Sigma_{i=1}^{10} \Sigma_{j=1}^{10} PMI(w_i,w_j)~,
\end{equation*}
where the PMI of a given pair of words $(w_i, w_j)$ is $PMI (w_i,w_j)
= log \frac{p(w_i,w_j)}{p(w_i)p(w_j)}$.  The average of the PMI score
over all the 10 topics is used as the final measure of the PMI score.
We use our own dataset to calculate the word probabilities used in
this score (rather than using the Wikipedia corpus \cite{baldwin10}).

\noindent \textbf{Held-out Probability:}

Another way of evaluating topic models is to compare predictive
performance by estimating the probability of a subset of held-out
documents.  We used the Left to Right evaluation algorithm as
described in \cite{wallach} to calculate these values, which is an
unbiased method.  Another approach is the so-called document
completion method \cite{wallach}, however with so few words we felt
holding out a subset of a (small) document was ill-advised.


\section{Results for Pooling Schemes}

In this section we discuss the results of the experimental evaluation
of the tweet pooling schemes introduced in
Section~\ref{sec:pooling}. The datasets used were described in
Section~\ref{sec:dataset} while the evaluation metrics used were
described in Section~\ref{sec:evaluation}.

\label{sec:init_results}

\subsection{Document Characteristics}

We first have a look at the document characteristics of the documents in the
different pooling schemes for the three datasets. Characteristics
like the number of documents affect LDA directly and hence it will be
interesting to look at what the training data consists
of. Table~\ref{tbl-3} presents the required statistics.

%\begin{figure*}
{
\begin{table*}%[!ht]
%\setcounter{table}{3}
\centering
\caption{Document Characteristics for different schemes}\label{tbl-3}
\resizebox{14cm}{!} 
{
	\begin{tabular}{|l|ccc|ccc|ccc|}
	\hline
	Pooling Scheme  & \multicolumn {3}{c|}{\#of docs} & \multicolumn {3}{c|}{Avg \# of words/doc} & \multicolumn {3}{c|}{Max \# of words/doc}\\
	\hline
	 & Generic & Specific & Events &  Generic & Specific & Events &  Generic & Specific & Events\\
	\hline
	Authorwise & 208300 & 118133 & 67387 & 17.6 & 20.4 & 15.4 & 4893 & 3586 & 2775 \\
	\hline
	Unpooled & 359478 & 214580 & 207128 & 10.2 & 10.9 & 9.7 & 35 & 49 & 32 \\
	\hline
	Burst Score & 7658 & 7436 & 5434 & 76.5 & 154.2 & 71.6 & 61918 & 420249 & 57794 \\
	\hline
	Hourly & 465 & 464 & 463 & 8493.4 & 5387.5 & 2422 & 20144 & 18869 & 38893 \\
	\hline
	Hashtag & 8535 & 7029 & 4099 & 70.4 & 187.2 & 78.4 & 61918 & 420249 & 57794 \\
	\hline
	\end{tabular}
}\vspace*{-6pt}
\end{table*}
}
%\end{figure*}

The statistics presented above highlight the differences in the
characteristics of the documents on which LDA models have been
trained. The number of documents decreases as we move from Unpooled
scheme to Authorwise and Hashtagwise pooling scheme, while the
corresponding size of the documents in each case increases. On an
average the document size increases by a factor of seven in hashtag-based
pooling when compared against unpooled or authorwise pooling
schemes. Thus each document in hashtag-based pooling contains more
content from which LDA could possibly extract latent semantics. On the
other extreme lies the temporal pooling with very less number of
documents and hence each document of a much larger size. Such large
documents might impact the topic model in an unpleasant manner. These
statistics highlights that hashtag-based pooling scheme lies mid-way
between both the extremes (small documents in unpooled tweets vs large
documents in temporal pooling) and hence suggests that hashtag-based
pooling should perform optimally in comparison to other schemes.

\subsection{Comparison of Pooling Schemes}

For the three datasets (viz. Generic, Specific and Events) and pooling
schemes, we next evaluate the Purity scores, NMI scores, PMI scores
and the Held-out probabilities in Table~\ref{tbl-456} on the topic
model obtained by training LDA using each scheme.

\begin{table*}%[!ht]
%\setcounter{table}{11}
\centering
\caption{Results of different pooling schemes}\label{tbl-456}
\resizebox{17cm}{!} 
{
	\begin{tabular}{|l|ccc|ccc|ccc|ccc|}
	\hline
	Scheme  & \multicolumn {3}{c|}{Purity} & \multicolumn {3}{c|}{NMI Score} & \multicolumn {3}{c|}{PMI score} & \multicolumn {3}{c|}{Log Held-out Probability}\\
	\hline
	 & Generic & Specific & Events &  Generic & Specific & Events &  Generic & Specific & Events & Generic & Specific & Events\\
	\hline
	Unpooled & $ 0.49\pm 0.08 $ & $ 0.64\pm 0.07 $ & $ 0.69\pm 0.09 $ & $ \textbf{0.28} \pm \textbf{0.04} $ & $ 0.22\pm 0.05 $ & $ 0.39\pm 0.07 $ & $ -1.27\pm 0.11 $ & $ 0.47\pm 0.12 $ & $ 0.47\pm 0.13 $ &  $-82.2\pm 6.3$ & $-89.3\pm 7.2$ & $-86.3\pm 7.4$\\
	\hline
	Author & $ \textbf{0.54} \pm \textbf{0.04} $ & $ 0.62\pm 0.05 $ & $ 0.60\pm 0.06 $ & $ 0.24\pm 0.04 $ & $ 0.17\pm 0.04 $ & $ 0.41\pm 0.06 $ & $ 0.21\pm 0.09 $ & $ 0.79\pm 0.15 $ & $ 0.51\pm 0.13 $ & $-63.0\pm 4.3$ & $-68.6\pm 4.7$ & $-66.4\pm 5.2$\\
	\hline
	Hourly & $ 0.45\pm 0.05 $ & $ 0.61\pm 0.06 $ & $ 0.61\pm 0.07 $ & $ 0.07\pm 0.04 $ & $ 0.09\pm 0.04 $ & $ 0.32\pm 0.05 $ & $ -1.31\pm 0.12 $ & $ 0.87\pm 0.16 $ & $ 0.22\pm 0.14 $ & $-64.8\pm 6.2$ & $-69.4\pm 5.8$ & $-67.9\pm 7.1$\\
	\hline
	Burstwise & $ 0.42\pm 0.07 $ & $ 0.60\pm 0.04 $ & $ 0.64\pm 0.06 $ & $ 0.18\pm 0.05 $ & $ 0.16\pm 0.04 $ & $ 0.33\pm 0.04 $ & $ 0.48\pm 0.16 $ & $ 0.74\pm 0.14 $ & $ 0.58\pm 0.16 $ & $-56.7\pm 5.5$ & $-59.0\pm 4.5$ & $-57.8\pm 6.1$\\
	\hline
	Hashtag & $ \textbf{0.54}\pm \textbf{0.04} $ & $ \textbf{0.68}\pm \textbf{0.03} $ & $ \textbf{0.71}\pm \textbf{0.04} $ & $ \textbf{0.28}\pm \textbf{0.04} $ & $ \textbf{0.23}\pm \textbf{0.03} $ & $ \textbf{0.42}\pm \textbf{0.05} $ & $ \textbf{0.78}\pm \textbf{0.15} $ & $ \textbf{1.43}\pm  \textbf{0.14} $ & $ \textbf{1.07}\pm\textbf{0.17} $ &  $\textbf{-55.9}\pm\textbf{ 4.3}$ & $\textbf{-58.9}\pm\textbf{ 4.1}$ & $\textbf{-55.4}\pm\textbf{ 4.3}$\\
	\hline
	\end{tabular}
}
\end{table*}

\begin{comment}
Why ``initial'' results? -- Wray
\end{comment}

Based on these results we conclude that hashtag-based pooling scheme
\emph{clearly} performs better than unpooled scheme as well as other
pooling schemes.


\begin{comment}
An obvious question to ask is: Can we do better? In
the next section we look into hashtag-based pooling in detail and
devise methods which further improve the results and provide better
topics.  We did not evaluate the held-out probability in later
experiments as it agreed generally with the other scores,
and is not as appropriate  as a quality metric for the
undirected information task.
\end{comment}




\section{Related Work}

\label{sec:related_work}
Topic modeling is widely used in text mining communities with LDA
being the benchmark.  LDA has been extended in a variety of ways, and
in particular for social networks and social media, a number of
extensions to LDA have been proposed.  For example, \cite{newman11}
proposed two methods to regularize the learning of topic models aimed
at short text snippets. While the focus of this work was on blogs and
search result snippets, it would be interesting to see how well they
work on Twitter data.  Also, the combination of the work proposed in
\cite{newman11} with the tweet pooling schemes we describe before
could produce interesting results. For automatic
 hashtag labeling that proved crucial to improving topics in our hastag-based pooling model, \cite{zangerle2011recommending} also uses tweet similarity as a
criteria, but does not explore metrics based on inverse author
frequency ~\cite{iaf} that we found to offer the most robust
performance across datasets and evaluation metrics. Additional features  for hashtag assignment can be found in the comprehensive study \cite{yang2012www} which can be leveraged in future extensions.

Our work is quite different from many pioneering studies on Twitter
and topic modeling because we focus on how we could get better
topic coherence over tweets with minimal modification to existing
models. Prior work on topic modeling for tweets includes the work of
\cite{ramage} which presents a scalable implementation of a partially
supervised learning model (Labeled LDA). % that maps the content of the Twitter feed into dimensions and characterizes users and tweets using this model.
 \cite{wayne} empirically compare the content of Twitter
with a traditional news medium, New York Times, using unsupervised
topic modeling. \cite{hong} use the topic modeling approach for
predicting popular Twitter messages and classifying Twitter users and
corresponding messages into topical categories. %\cite{han2012tist} propose a  novel method for normalising ill-formed out-of-vocabulary words in short  microblog messages.
The {T}witterRank system~\cite{Weng2010wsdm} and \cite{hong} uses author-based pooling to apply LDA to tweets. \cite{wayne} compared
topic characteristics between twitter and traditional news media; they propose to use one topic per tweet (similar to
PLSA), and argues that this is better than no
pooling, or the author-topic model. %\cite{kireyev2009} used term weighting to tackle term sparseness in LDA, the weights are derived from LSA vector length.
 \cite{Naveed2011cikm} used LDA for tweet retrieval. In
addition, they used retweet as an indicator of "interestingness" to
improve retrieval quality, which suggests additional features we
could incorporate in future extensions to our pooling framework.

Our work is different from these in the sense that we provide a simple
yet effective way which greatly improves the quality of topics
obtained without making any major complicated modifications to
standard LDA. The detailed experiments on a variety of datasets
highlight our novel contribution of hashtag-based pooling \begin{comment}and automatic hashtag labeling using similarity metrics like IAF~\cite{iaf} \end{comment} as an approach that improves a range of topic coherence measures.

\section{Summary and Conclusion}

\label{sec:conclusion}

%The work described in t
This paper presents a way of aggregating tweets in order to improve performance of topic models in terms of quality of topics obtained measures by the ability to reconstruct clusters and topic coherence. The results presented in Table~\ref{tbl-456} suggest that hashtag-based pooling outperforms all other pooling strategies including the default way of training topic models on Twitter data (unpooled).

\begin{comment}
Since a major portion of Twitter data does not contains hashtags we looked at ways of assigning hashtags to tweets. Insights from Hashtag Assignment results (Table~\ref{tbl-9}) suggest that when the main aim is to use the topics obtained to extract different events mentioned in the Twitter data, one should use hashtag assignment with a relaxed threshold($\sim0.5$). The high values of Purity scores and NMI values for low threshold support this claim.

When the goal is to obtain interesting topics with topic words
pertaining to the same common theme (coherent topic words), hashtag
assignment with strict constraints (threshold $\sim 0.9$) works
well. The PMI scores in Table~\ref{tbl-9} highlight that topic
coherence increases down the column with a threshold of
0.9 giving most coherent topics. Overall, results shown in Table~\ref{tbl-10} compare unpooled scheme with simple hashtag pooling and hashtag assignment schemes.
\end{comment}


Across diverse datasets and various topic coherence metrics, \begin{comment}the best\end{comment} hashtag assignment method performs substantially better in comparison to an unmodified LDA baseline and a variety of existing and novel
pooling schemes. This indicates the promise of this novel automatic
hashtag \begin{comment}labeling and \end{comment}pooling approach that drastically improves the quality of topic models for Twitter and microblog data.

%To this end, we have achieved our initial goal.  We have introduced
%novel techniques for automatic tweet hashtag labeling and pooling that
%have produced topics representative of the events queried to construct
%the source data and have substantially improved various quantitative
%measures of topic coherence over a wide range of data and topic
%coherence measures.  Further, we have achieved this without modifying
%the LDA algorithm itself but rather by preprocessing its input.

%This technique worked well across a diverse set of data and suggests
%that continued investigation into aggregation techniques for topic
%modeling on short text and microblog documents is warranted given its
%highly successful outcome demonstrated in this work.

%\section*{Acknowledgments} 
%
%NICTA is funded by the Australian
%Government as represented by the Department of Broadband,
%Communications and the Digital Economy and the Australian Research
%Council through the ICT Centre of Excellence program.

%'apalike-fr' style below applies smallcaps style on author names
%in order to apply 'apalike-fr' the babel package must be given [frenchb] option instead of [english]
% \usepackage[frenchb]{babel} also causes title "References" to render with French accents like "R\'ef\'erences"
%\bibliographystyle{apalike-fr}

%'apa' style does not apply "smallcaps style" on author names and goes with the [english] option in the babel package

%\bibliography{colingbiblio}
%\bibliographystyle{aaai}

%\bibliographystyle{aaai}
\bibliographystyle{abbrv}
\bibliography{colingbiblio}
%\bibliography{sig-alternate}
% \nocite{*}
%\bibliographystyle{coling2012

%%================================================================
\end{document}
